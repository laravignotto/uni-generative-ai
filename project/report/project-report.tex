\documentclass[a4paper,11pt]{article}

\usepackage[top=2.4cm,
            bottom=2.4cm,
            left=2.8cm,
            right=2.8cm]{geometry}
\usepackage[english]{babel}
\usepackage[utf8]{inputenc}
\usepackage[T1]{fontenc}
\usepackage{amsmath}
% \usepackage{mathtools}
\usepackage{bm}
% \usepackage{makeidx}
\usepackage[italian]{varioref}
\usepackage{hyperref}
\usepackage{url}
% \usepackage{wasysym}
% \usepackage{amssymb}
% \usepackage{pifont}
\usepackage[font=small,labelfont=bf]{caption}
% \usepackage{subcaption}
\usepackage[dvipsnames]{xcolor}
% \usepackage{nicefrac}
% \usepackage{cancel}
% \usepackage{tikz}
% \usetikzlibrary{arrows.meta,calc,decorations.markings,math,arrows.meta}
% \usetikzlibrary{positioning}
% \usetikzlibrary{trees}
% \usetikzlibrary{shapes.geometric}
% \usetikzlibrary{decorations.pathmorphing}
% \usetikzlibrary{tikzmark}
\usepackage{tocloft}
% Center the toc index title
\renewcommand\cfttoctitlefont{\hfill\Large\bfseries}
\renewcommand\cftaftertoctitle{\hfill\mbox{}}
% Adjust the width between the titles and page numbers
\cftsetindents{section}{3cm}{2em} % Adjust '2em' to your preferred spacing
\cftsetindents{subsection}{4cm}{2em}
% If you also want to center the titles, you can use the following commands
\renewcommand{\cftsecleader}{\cftdotfill{\cftdotsep}}
\renewcommand{\cftsecafterpnum}{\hspace*{3cm}\mbox{}\par}
\renewcommand{\cftsubsecleader}{\cftdotfill{\cftdotsep}}
\renewcommand{\cftsubsecafterpnum}{\hspace*{3cm}\mbox{}\par}


\title{\Huge Project Report}
\author{Vignotto Lara -- 111794}
\date{\today}


\begin{document}

\maketitle
\vspace{1cm}
\tableofcontents
\vspace{3cm}



%%%%%%%%%%%%%%%%%%%%%%
\section{Introduction}
In this report, I will illustrate how I generated a short comic book story exploiting free online generative AI tools. The document is divided into the following sections:
\begin{itemize}
    \item Tools and Methods: I will describe the AI tools and methods I used to generate the story;
    \item The Story -- Plot and Text: I will present the story, how I generated the text, and the limitations I encountered;
    \item The Art -- Images: I will describe how I generated the illustrations and the limitations and biases of the AI tools I used;
    \item Conclusions: I will draw some conclusions and discuss the results.
\end{itemize}
At the end of this document, there are two appendices: the first one contains the original prompts I used to generate the story, and the second one contains the original prompts I used to generate the illustrations.



%%%%%%%%%%%%%%%%%%%%%%
\section{Tools and Methods}
I used two generative AI tools to generate the story and the illustrations. For the story, I used ChatGPT 3.5 \cite{gpt}, a conversational AI model developed by OpenAI. For the illustrations, I used DALL-E 3 integrated into Microsoft Copilot \cite{copilot2023}, a generative model developed by OpenAI and Microsoft. 
ChatGPT is an AI language model based on OpenAI's GPT (Generative Pre-trained Transformer) architecture. DALL-E 3 is also built upon OpenAI's GPT architecture, but it is specialized for image generation tasks. 

I initially also tried to use other generative AI tools, such as Google Bard \cite{bard} and Microsoft Copilot to generate the story, and DreamStudio \cite{dreamstudio} to generate the illustrations. However, I found that ChatGPT and DALL-E 3 were the most effective tools for my purposes. I will talk about the limitations of these tools in the next sections.

To paginate the comic book, I used \LaTeX, a typesetting system that is widely used for technical and scientific documents. I used the \texttt{tikz} package to create the comic book layout and the \texttt{graphicx} package to include the images.

All the code I used to write this report and paginate story and the illustrations is available on my GitHub repository \cite{github}.



%%%%%%%%%%%%%%%%%%%%%%
\section{The Story -- Plot and Text}

The story is a science fiction tale set in a future where humanity lives in an arcology, a massive, self-sustaining structure that houses communities and ecosystems. The protagonist is Dr.~Evelyn Hayes, a communication scientist devoted to her work within the arcology. One day, she receives a mysterious message from an unknown source. Her young daughter seems to possess an intuition that the message is a cry for help from a distant planet. Dr.~Hayes decides to investigate the message and embarks on a journey to the planet. 

The story is intended as a pilot episode, and could be expanded into a series. The title, \emph{Echoes of Hope -- A Journey Beyond The Stars}, was also generated by ChatGPT. The comic book is divided into 12 pages, each containing 1 to 5 panels.

\subsection*{Methodology} 
I started by asking ChatGPT to generate a plot for a science fiction comic book. 
\begin{quote}
    \textbf{Me:} Hi! I want to write a small story with an open ending (a pilot episode). The story is set inside the arcology humanity built as a last resort chance to perpetuate the human race. The main character could be a woman whose job is trying to communicate with long-distance computers or aliens, despite them being in deep space. Any suggestions?
\end{quote}

\subsection*{Limitations}



%%%%%%%%%%%%%%%%%%%%%%
\section{The Art -- Images}

\subsection*{Methodology}

\subsection*{Limitations}
% examples
% PROMPT-1: A girl of about 6 years old with a cherubic face framed by wisps of chestnut hair that cascade in loose curls around her shoulders sits in her quarters in an arcology, surrounded by toys and drawings scattered across the floor. She is deeply engrossed in drawing mysterious symbols on a sheet of paper, her brow furrowed in concentration as she works.
% PROMPT-2: Comic book style. A girl of about 6 years old with a cherubic face framed by wisps of chestnut hair that cascade in loose curls around her shoulders sits in her quarters in an arcology, surrounded by toys and drawings scattered across the floor. She is deeply engrossed in drawing mysterious symbols on a sheet of paper, her brow furrowed in concentration as she works.



%%%%%%%%%%%%%%%%%%%%%%
\section{Conclusions}





\newpage
\appendix

%%%%%%%%%%%%%%%%%%%%%%
\section{Appendix: Text Prompts}



%%%%%%%%%%%%%%%%%%%%%%
\newpage
\section{Appendix: Image Prompts}


\newpage 
\bibliographystyle{plain} 
\bibliography{references} 


\end{document}